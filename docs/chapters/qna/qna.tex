\chapter{Q\&A}\label{ch:q&a}


\section{Where is the configuration files?}
As a convention, all the files are placed in \ttt{config/fuji/} directory.


\section{What is .json file?}
A json file is a text file, whose name normally ends with \tbf{.json}.


\section{How can I edit a configuration file?}
To ensure the \ttt{readable} and \ttt{transparent}, most of the files are saved as \ttt{pure text format}.
You can open them with a \ttt{text editor}.

\begin{tips}{Use a moderen text editor.}
    Some files may have a large number of lines, so it's highly recommended to use a \ttt{modern} text editor, which can highlight symbols and reveal the structure of the file, such as:
    \begin{enumerate}
        \item \href{https://code.visualstudio.com/}{Visual Studio Code}
        \item \href{https://vscode.dev/}{Visual Studio Code - Web Online Editor}
        \item \href{https://www.vim.org/}{Vim}
        \item \href{https://www.gnu.org/software/emacs/}{Emacs}
        \item \href{https://www.sublimetext.com/}{Sublime Text}
    \end{enumerate}
\end{tips}


\section{What is .dat file?}
The file whose name ends with \tbf{.dat} are the \ttt{vanilla minecraft NBT format file}.
To open such a file, you need to use a \ttt{NBT Editor}, such as \ttt{NBTExplorer}.


\section{How to update fuji to a new version?}

\subsection{Backup the data}
Back up the \ttt{config/fuji} directory.

\subsection{Test the new version in your test-environment}
Put the new version of fuji into \ttt{mods/} directory, start the server, and adjust the configuration to what you want.

\begin{warn}{Don't test new changes directly in your production-environment}
    It's highly recommended to setup a test-environment for a network-maintainer, so that you can test and tweak installed mods into what you want, and avoid un-expected situations.
\end{warn}

\subsection{Apply the changes to production-environment}
Now, it's ready to apply the changes to your production-environment.


\section{Fuji conflits with one of my mods.}
You can \ttt{disable} the \ttt{conflicting module}.
If possible, create an issue at \issueurl, so that we can solve this later.


\section{How can I report bugs or suggest new features?}
You can create an issue at \issueurl



