\LevelOne{echo}
This module provides commands to send echo to players.

\begin{note}{Many other modules require this moudle}
    Other modules may generate a \ttt{default configuration} including the commands provided by \ttt{echo} module.
    If the \ttt{echo module} is disabled, then these \ttt{echo commands} will not exist, causing a \ttt{command syntax error} while executing these commands.
\end{note}

\LevelTwo{Sub-Module}
\LevelThree{send-message}
\fcmd{/send-message}


\begin{example}{Say hello to a player}
    \cmd{/send-message \player Hello \ph{player:name}}
\end{example}

\LevelThree{send-broadcast}
\fcmd{/send-broadcast}


\begin{example}{Say hello to all players}
    \cmd{/send-broadcast Hello \ph{player:name}}
\end{example}

\LevelThree{send-actionbar}
\fcmd{/send-actionbar}


\begin{example}{Say hello to a player}
    \cmd{/send-actionbar \player Hello \ph{player:name}}
\end{example}

\LevelThree{send-title}
\fcmd{/send-title}

\begin{example}{Send title to a player}
    \cmd{/send-title \player --mainTitle "<rainbow>Hello" --subTitle "<blue>World" --fadeInTicks 60 --stayTicks 60 --fadeOutTicks 60}
\end{example}

\begin{example}{Send title to online players}
    \cmd{/foreach send-title \ph{player:name} --mainTitle "<rainbow>Hello \ph{player:name}"}
\end{example}
\LevelThree{send-toast}
\fcmd{/send-toast}


\begin{example}{Send toast to a player}
    \cmd{/send-toast \player --icon minecraft:golden\_carrot <rb>eat this carrot}
\end{example}

\LevelThree{send-chat}
\fcmd{/send-chat}


\begin{example}{Send chat as a player}
    \cmd{/send-chat Steve i am steve.}
\end{example}

\begin{example}{Send chat as a player for online players}
    \cmd{/foreach send-chat \ph{player:name} i am \ph{player:name}}
\end{example}

\LevelThree{send-bossbar}
\fcmd{/send-bossbar}

\begin{example}{A simple exapmle}
    \cmd{/send-bossbar \player Hello World}
\end{example}

\begin{example}{All in one exapmle}
    \cmd{/send-bossbar \player --stepType BACKWARD --totalMs 5000 --color PURPLE --style NOTCHED\_6 --notifyMeOnComplete true --commandList "say the player \ph{player:name} is healed|heal others \ph{player:name}" <rb>Healing is coming [elapsed\_time]/[total\_time]/[left\_time]}
\end{example}

\LevelThree{send-custom}
\fcmd{/send-custom}

\begin{example}{Create a custom text}
    Create a plain text file named \str{guide} in \textbf{config/fuji/modules/echo/send\_custom/custom-text/guide} with content:
    \begin{textcode}
        <blue>===== custom text =====
        Hello <orange>%player:name%</orange>, you are in <orange>%world:name%</orange> now.

        <hover:show_text:'you see me!'>Hover me</hover>

        <click:run_command:'/back'>click me to run `/back` command</click>

        <u><i><click:change_page:'3'>click me to the third page (this only works inside a book)</click></i></u>

        <newpage><blue>This is the second page!

        <click:suggest_command:'/back'>click me to suggest /back command (This doesn't work inside a book)</click>

        <insert:'hello'>shift + click me to insert "hello" (This doesn't work inside a book)</insert>

        <click:open_url:'https://placeholders.pb4.eu/user/text-format/'>click me to open the url</click>

        <newpage>This is the third page!

        <bold><click:change_page:'1'>click me to the first page</click></bold>

        <orange>You can press `<keybind:'key.jump'>` key to jump!</orange>

        <gradient:red:green:blue>This is gradient text.</gradient>

        <rb>The rainbow text</rb>

        <newpage>The end.
    \end{textcode}
\end{example}

\begin{example}{Send custom text as a book}
    \cmd{/send-custom as-book \player guide --author "alice" --title "<rb>The Guide" --giveBook true --openBook true}
\end{example}

\begin{example}{Send custom text as a message}
    \cmd{/send-custom as-message \player guide}
\end{example}
