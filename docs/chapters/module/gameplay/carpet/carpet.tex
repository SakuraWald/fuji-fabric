\LevelThree{carpet}

\LevelFour{Sub-Module}
\LevelFive{fake\_player\_manager}

\subsubparagraph{Purpose}
Enable this module requires carpet-fabric mod installed.
This module provides some management for fake-player.

\subsubparagraph{Purpose2}
\subsubsubparagraph{first}
\subsubsubparagraph{second}
\subsubsubsubparagraph{deep1}
\subsubsubsubparagraph{deep2}

\subsubparagraph{Purpose3}
\subsubsubparagraph{third}
\subsubsubsubparagraph{deep3}


Command\\
/player who = query the owner of the fake-player.\\
/player renew = renew all of your fake-players.

Configuration\\
caps\_limit\_rule\\
How many fake-player can each player spawn (in different time)?
The tuple means (day\_of\_week, minutes\_of\_the\_day, max\_fake\_player\_per\_player).
The range of day\_of\_week is [1,7].
The range of minutes\_of\_the\_day is [0, 1440].
For example: (1, 0, 2) means if the days\_of\_week >= 1, and minutes\_of\_the\_day >= 0, then the max\_fake\_player\_per\_player now is 2.
Besides, you can add multi rules, the rules are checked from up to down.
The first rule that matches current time will be used to decide the max\_fake\_player\_per\_player.
You can issue \ic{/player who} to see the owner of the fake-player.
Only the owner can operates the fake-player. (Op can bypass this limit)

renew\_duration\_ms\\
How long should we renew when a player issue \ic{/player renew}
The command /player renew allows the player to manually renew all of his fake-player.
If a fake-player don't gets renew, then it will expired and get killed.
Use-case: to avoid some long-term alive fake-player.

transform\_name\\
The rule to transform the name of fake-player.
Use-case: add prefix or suffix for fake-player.

use\_local\_random\_skins\_for\_fake\_player\\
Should we use local skin for fake-player?
Enable this can prevent fetching skins from mojang official server each time the fake-player is spawned.
This is mainly used in some network situation if your network to mojang official server is bad.

\LevelFive{better\_info}


\subsubparagraph{Purpose3}

Purpose\\
Add nbt query for /info block command.
Add the command /info entity.



