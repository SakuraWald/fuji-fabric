\LevelOne{command\_permission}

\LevelTwo{Purpose}
This module provides the customization of \tbf{the requirement of all commands}.

\LevelTwo{Command}
\LevelThree{/command-permission}

\LevelTwo{How it works}
The vanilla minecraft use a command system named brigadier.\\
All the commands are registered, parsed and executed by brigadier. \\
In this system, all commands are build into \tbf{a tree structure}, that is to say, all commands are a direct or in-direct child of the \tbf{root command node}.

\begin{example}{What is the path of a specific command node?}
    For example, the command \ic{/gamemode creative Steve} is composed by 3 command node:
    \begin{description}
        \item [literal command node] = \ttt{"gamemode"}
        \item [argument command node] = \tbf{a gamemode argument}
        \item [argument command node] = \tbf{a player argument}
    \end{description}
    We say that the command path of \ic{/gamemode creative Steve}, is \ttt{["gamemode", "gamemode", "target"]}.
\end{example}

Also, each \tbf{command node} has its \tbf{requirement}, which is a \tbf{predicate} to check if the \tbf{command source} can use the command node.

\begin{tips}{Query all the registered command path}
    \ic{/lp group default permission set fuji.permission\ldots}
\end{tips}

\LevelTwo{Example}
\begin{example}{Allow everyone to use /op command}
    \ic{/lp group default permission set fuji.permission.op true}
\end{example}

\begin{example}{Allow the client-side to use gamemode switcher menu}
    After you assign the \ttt{/gamemode} command permission for players, the client-side also requires to install a mod to bypass the client-side permission checking: \url{https://modrinth.com/mod/switcher}
\end{example}

