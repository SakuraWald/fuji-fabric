\LevelOne{chat}
\LevelTwo{Purpose}
This module provides chat system customization.

\LevelTwo{Feature}
\begin{enumerate}
    \item You can create your own \ttt{regex transformer} to replace \ttt{chat message} sent by players.
    \item You can insert any \ttt{placeholder} like \ph{world:name}
    \item {You can insert player's prefix and suffix. Just insert \ph{fuji:player\_prefix} and \ph{fuji:player\_suffix}.}
    \item You can insert \str{item}, \str{inv} and \str{ender} to display your item, inventory and ender-chest.
    \item You can insert \str{Steve} to mention another player named Steve.
    \item You can insert \str{pos} to show the position.
    \item You can use \ttt{markdown language} to define simple format.
    \item You can use \ttt{mini-message language} to define complex format.
    \item Besides the \ttt{server chat format}, each player can also set their per-player chat format.
    \item This module doesn't \tbf{cancel} the vanilla chat events, so it can work with other chat related mods.
\end{enumerate}

\begin{note}{Set prefix and suffix for players}
    Luckperms is required to set \ttt{prefix} and \ttt{suffix}. \\
    \\
    After you installed \ttt{luckperms} mod, just issue \ic{/lp group default meta setprefix <yellow>[awesome]} to assign prefix. \\
    Don't forget to change the format in chat module, and issue \ic{/fuji reload}
\end{note}

\begin{tips}{Write complex style using mini-language}
    You can use \ttt{mini-language} to write complex text.\\
    \\
    See more:
    \begin{enumerate}
        \item \url{https://docs.advntr.dev/minimessage/format.html}
        \item \url{https://placeholders.pb4.eu/user/quicktext}
    \end{enumerate}
\end{tips}

\LevelTwo{Command}
\LevelThree{/chat}

\LevelTwo{Placeholder}
\begin{description}
    \item[\ph{fuji:pos}] player current position
    \item[\ph{fuji:player\_prefix}] player prefix
    \item[\ph{fuji:player\_suffix}] player suffix
\end{description}

\LevelTwo{Configuration}

\LevelThree{format}
The server chat format for all players.

\LevelFour{rewrite}
The \ttt{regex language} list used to \ttt{rewrite} the player chat message.\\
You can use regex language to transform player's chat input (only chat message, no command usage).

\LevelFour{mention\_player}
If you insert \tbf{Steve} in chat message, then the player named Steve will get audio mention.

\LevelFive{sound}

The type of \tbf{sound} used to notify the mentioned player.
\begin{tips}{Query all identifiers of sound}
    \ic{/playsound \ldots}
\end{tips}

\LevelFive{volume}

\LevelFive{pitch}

\LevelFive{repeat\_count}

The sound repeat count.

\LevelFive{interval\_ms}

The interval between each repeat.

\LevelFour{spy}
Log chat information into server-console.

\LevelTwo{Sub-Module}
\LevelThree{display}

\LevelFour{Placeholder}
\begin{description}
    \item [\ph{fuji:item}] display player's main-hand item.
    \item[\ph{fuji:inv}] display player's inventory.
    \item[\ph{fuji:ender}] display player's ender-chest.
\end{description}

\LevelFour{Configuration}
\begin{Configuration}
    \item[expiration\_duration\_s]{
        For each display data, how long should we save in the memory.
        Note that if a player shares its inventory items, then fuji will save a copy of his inventory data in the memory.
    }

\end{Configuration}
\LevelThree{history}
New joined players can see the historical chat messages.

\LevelFour{Configuration}
\LevelFive{buffer\_size}

How many chat messages should we save, so that we can send for a new-joined player.






\LevelTwo{Reference}
\begin{enumerate}
    \item \href{https://placeholders.pb4.eu/user/default-placeholders/}{Text Placeholder API - default placeholders}
    \item \href{https://luckperms.net/wiki/Prefixes,-Suffixes-&-Meta}{Luckperms - prefix, suffix and meta}
\end{enumerate}